% !TeX root = ../presentation.tex

\setFooterToMager

\section{Quanten-Computing}

\begin{frame}{Quanten-Computing $\blacktriangleright$ Grundidee \cite{nielsen2010quantum}}
    \begin{columns}
        \begin{column}{0.55\textwidth}
            \begin{itemize}
                \item Qubits statt klassischer Bits
                \item Superposition: gleichzeitiger Zustand von 0 und 1
                \item Verschränkung zwischen Qubits
                \item Gemeinsame Operationen auf vielen Zuständen
                \item Interferenz verstärkt korrekte Ergebnisse
            \end{itemize}
        \end{column}

        \begin{column}{0.45\textwidth}
            \begin{figure}
                \centering
                \includegraphics[width=\textwidth]{./assets/QBitsZustand.png}
                \caption{\footnotesize Zustand eines Qubits~\cite{wikipediaQubitBlochKugel}}
            \end{figure}
        \end{column}
    \end{columns}
\end{frame}



\begin{frame}{Quanten-Computing $\blacktriangleright$ Aktueller Stand der Technik \cite{nielsen2010quantum, gidney2021factor}}
    \begin{itemize}
        \item RSA und ECC gelten heute als sicher, aber gefährdet durch Quantencomputer
        \item Spezielle Quantenprogramme bereits schneller als klassische Supercomputer
        \item Geschwindigkeitsvorteile bis zu $13{.}000\times$~\cite{google2025observation}
        \item Experimentelle Demonstrationen existieren 
        \item Noch keine praktischen Angriffe auf Kryptosysteme 
        \item Quantenhardware aktuell stark limitiert 
    \end{itemize}
\end{frame}



\begin{frame}{Quanten-Computing $\blacktriangleright$ Algorithmischer Kern \& Ausblick \cite{gidney2021factor, roetteler2017quantum}}
    \begin{itemize}
        \item Shors Algorithmus als zentrale Bedrohung 
        \item Nutzung von Quanten-Parallelität
        \item Sehr hoher Bedarf an fehlerkorrigierten Qubits  % TODO grafk zu korigierten qbits
        \item Post-Quanten-Kryptografie langfristig notwendig 
    \end{itemize}
\end{frame}


\begin{frame}{Quanten-Computing $\blacktriangleright$ Quanten-Sicherheit in Signal \footnotesize\cite{connellSignalProtocolPostQuantum2025, gentyalaQuantumResistanceSignal2025}}
    Schutz vor \emph{Harvest Now, Decrypt Later} Angriffen

    \begin{itemize}
        \item[I.] Absicherung des initialen Handshakes (X3DH) mit \alert{PQXDH}
        \begin{itemize}
            \item Hybrider Ansatz: Elliptic Curve + \alert{CRYSTALS-Kyber}
        \end{itemize}
        
        \pause
        \item[II.] Erweiterung von \emph{Forward Secrecy} (FS) und \emph{Post-Compromise Security} (PCS)
        \begin{itemize}
            \item Double Ratchet + \alert{SPQR} $\rightarrow$ Triple Ratchet
        \end{itemize}
    \end{itemize}
\end{frame}
