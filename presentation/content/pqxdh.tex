% !TeX root = ../presentation.tex

\setFooterToSchramm

\section{Post-Quantum Extended Diffie-Hellman (PQXDH)}

\begin{frame}{PQXDH $\blacktriangleright$ Einführung}
    \begin{itemize}
        \item X3DH anfällig gegenüber Quanten-Computing-Attacken~\cite{kretQuantumResistanceSignal2023,schmidtAnalysisSignalsPQXDH2024}

        \item Ziel: Verhinderung von \emph{Harvest Now, Decrypt Later} Attacken (HNDL)

        \pause

        \item Hinzufügen von PQ-Sicherheit zum \emph{Handshake}~\cite{schmidtAnalysisSignalsPQXDH2024}\\
            $\Rightarrow$ Erweiterung von X3DH~\cite{bhargavanAnalysisSignalsPQXDH2023, kretQuantumResistanceSignal2023}
        
        \item Kombination von bisherigem Ecliptic Curve (EC) und CRYSTALS-Kyber
        \note<2>{
            Kombination
            \begin{itemize}
                \item Einfügen eines Quanten-sicheren \emph{Shared Keys}, vgl. CRYSTALS-Kyber
                \item Kombination mit klassischem X3DH $\to$ beide Systeme müssen geknackt werden
            \end{itemize}
        }
        
        \item[$\Rightarrow$] \emph{Post-Quantum Extended Diffie-Hellman} (\alert{PQXDH})
    \end{itemize}
\end{frame}


\begin{frame}{PQXDH $\blacktriangleright$ Rollen im Key Agreement \footnotesize\cite{kretPQXDHKeyAgreement2024}}
    \begin{itemize}
        \item \textbf{Alice}: Senden von initialen, verschlüsselten Daten an Bob, Etablierung einer verschlüsselten Kommunikation
        
        \item \textbf{Bob}: Empfangen von Nachrichten (bspw. von Alice) zur Etablierung einer verschlüsselten Kommunikation, ggf. offline
        
        \item \textbf{Server}: Bereitstellung von benötigten Daten für Etablierung; Speicherung von Nachrichten von Alice an Bob zum späteren Abruf
    \end{itemize}

    \note{
        eigentlich aus vorherigem Vortrag
        \begin{itemize}
            \item Alice: Senden von initialen, verschlüsselten Daten an Bob
            \begin{itemize}
                \item Etablierung eines Shared Key → Verwendung für bidirektionale Kommunikation
            \end{itemize}
            
            \item Bob: möchte es Parteien wie Alice erlauben, dass Shared Key etabliert und verschlüsselte Daten gesendet werden
            \begin{itemize}
                \item ggf. offline, wenn Alice dies versucht → Beziehung zu einem Server
            \end{itemize}
            
            \item Server: Speicherung von Nachrichten von Alice an Bob; späterer Abruf durch Bob
            \begin{itemize}
                \item Veröffentlichung von Daten von Bob, welche Alice zur Verfügung gestellt werden
            \end{itemize}

        \end{itemize}
    }
\end{frame}


\begin{frame}{PQXDH $\blacktriangleright$ Design-Idee \footnotesize\cite{bhargavanAnalysisSignalsPQXDH2023, kretQuantumResistanceSignal2023,kretPQXDHKeyAgreement2024}}
    \begin{itemize}
        \item Erweiterung von X3DH um \emph{PQKEM Encapsulated Shared Secret} ($SS_{\text{KEM}}$)
        
        \item $SS_{\text{KEM}}$ und Diffie-Hellman-Werte $DH_{\{1,\ldots,4\}}$ aus X3DH als Input für \emph{Key Derivation Function} (KDF)
        
        \item Kombination der Outputs beider Algorithmen $\to$ Umgehen beider Systeme notwendig
    \end{itemize}

    \begin{figure}
        \includegraphics[height=2cm]{assets/pqxdh_kdf.drawio.pdf}
    \end{figure}
\end{frame}


\begin{frame}{PQXDH $\blacktriangleright$ Ablauf \footnotesize\cite{kretPQXDHKeyAgreement2024}}
    \begin{enumerate}[I.]
        \item Veröffentlichung der Keys (Bob $\to$ Server)
        \item Senden der initialen Nachricht (Alice $\to$ Bob)
        \item Empfang der initialen Nachricht (Bob)
    \end{enumerate}
\end{frame}


\begin{frame}{PQXDH $\blacktriangleright$ I. Veröffentlichung der Keys \footnotesize\cite{kretPQXDHKeyAgreement2024}}
    Bob veröffentlicht folgende Keys über den Server:

    \begin{tabular}{|l|l|}
        \hline
        Ecliptic Curve & PQKEM \\
        \hline\hline
        
        $SPK_B$: signierter Prekey & $PQSPK_B$: signierter \emph{Last Resort} Prekey \\
        $OPK_B$: One-Time-Prekey & $PQOPK_B$: signierter One-Time-Prekey \\
        $IK_B$: Identity Key & \\
        \hline
    \end{tabular}

    sowie Signaturen und Identifier

    \note{
        \begin{itemize}
            \item ähnliche Keys zu X3DH
            \item neue Äquivalente für PQKEM
            \item $PQSPK_B$ als \textbf{Last Resort}
        \end{itemize}
    }
\end{frame}


\begin{frame}{PQXDH $\blacktriangleright$ II. Initiale Nachricht senden \footnotesize\cite{bhargavanAnalysisSignalsPQXDH2023,kretPQXDHKeyAgreement2024,schmidtAnalysisSignalsPQXDH2024}}
    \begin{figure}
        % TODO: Legende einfügen
        \includegraphics[height=6cm]{assets/pqxdh_complete_alice.drawio.pdf}
    \end{figure}

    \note{
        \begin{itemize}
            \item Gesamt-Prozess für Phase II. \& III.
            \item Prekey-Bundle $\to$ PQXDH-Berechnung $\to$ initiale Nachricht
            \item gehen Schritt für Schritt durch, fangen bei Prekey Bundle an (zeigen)
        \end{itemize}
    }
\end{frame}


\begin{frame}{PQXDH $\blacktriangleright$ II. Initiale Nachricht senden \footnotesize\cite{bhargavanAnalysisSignalsPQXDH2023,kretPQXDHKeyAgreement2024,schmidtAnalysisSignalsPQXDH2024}}
    \begin{columns}
        \begin{column}{0.5\textwidth}
            \begin{itemize}
                \item Alice fragt \emph{Prekey Bundle} vom Server ab
                \item Server löscht One-Time-Prekeys (OTP)\\ $\to$ \emph{single use}
                \item Berücksichtigung falls keine OTP mehr verfügbar sind ($OPK_B$ bzw. $PQOPK_B$)
                \item Verifikation der Signaturen der Prekeys
            \end{itemize}
        \end{column}

        \begin{column}{0.5\textwidth}
            \begin{figure}
                \includegraphics[height=6cm]{assets/pqxdh_prekey_bundle.drawio.pdf}
            \end{figure}
        \end{column}
    \end{columns}

    \note{
        \begin{itemize}
            \item \emph{PQXDH Key Agreement}: Bobs Prekeys notwendig $\to$ Abfrage beim Server
            \item PQKEM One-Time Signed Prekey $PQOPK_B$ falls vorhanden
            \begin{itemize}
                \item falls nicht, dann \emph{Last Resort Signed Preky} $PQSPK_B$
                \item entsprechender Key: $PQ\boldsymbol{P}K_B$
                \item später: falls keine OTP verfügbar, im Protokoll berücksichtigen
            \end{itemize}
            \item Server löscht OTPs anschließend $\to$ \emph{single use}
            \item Alice verifiziert Signaturen der Prekeys; Abbruch bei Fehlschlag
        \end{itemize}
    }
\end{frame}


\begin{frame}{PQXDH $\blacktriangleright$ II. Initiale Nachricht senden \footnotesize\cite{bhargavanAnalysisSignalsPQXDH2023,kretPQXDHKeyAgreement2024,schmidtAnalysisSignalsPQXDH2024}}
    \begin{figure}
        \includegraphics[height=6cm]{assets/pqxdh_alice_dh_pqkem.drawio.pdf}
    \end{figure}

    \note{
        \begin{itemize}
            \item Generierung eines Ephemeral EC Schlüsselpaares mit Public Key $EK_A$,\\
                Berechnung DH-Werte (vgl. X3DH)
            \item wie zuvor: falls Prekey-Bundle keinen $OPK_B$ (EC OTP) hat, dann kein $DH_4$
            \item NEU: Generierung eines \emph{Shared Secrets} $SS_{\text{KEM}}$ und Ciphertext $CT_{\text{KEM}}$ aus $PQPK_B$
        \end{itemize}
    }
\end{frame}


\begin{frame}{PQXDH $\blacktriangleright$ II. Initiale Nachricht senden \footnotesize\cite{bhargavanAnalysisSignalsPQXDH2023,kretPQXDHKeyAgreement2024,schmidtAnalysisSignalsPQXDH2024}}
    \only<beamer>{
        % \framezoom(upper left x,upper left y)(zoom area width,zoom area depth)
        \framezoom<0><2>[border](1.7cm,2.35cm)(8cm,3cm)
        \framezoom<0><3>[border](1.7cm,3cm)(8cm,3cm)
        \framezoom<0><4>[border](5.2cm,3.5cm)(7cm,2cm)
    }

    \begin{figure}
        \includegraphics[height=6cm]{assets/pqxdh_complete_alice.drawio.pdf}
    \end{figure}

    \note<1>{
        Prekey-Bundle und DH-/KEM-Berechnung einordnen, nächste Folie: reinzoomen
    }
    
    \note<2>{
        \begin{itemize}
            \item berechnete Werte $DH_{\{1,\ldots,4\}}$ und $SS_{\text{KEM}}$ in KDF $\Rightarrow$ Output: \emph{Shared Key} $SK$
            \begin{itemize}
                \item falls kein $OPK_B$ und damit kein $DH_4$, dann hier weglassen
            \end{itemize}
        \end{itemize}
    }

    \note<3>{
        \begin{itemize}
            \item Berechnung der \emph{Associated Data Bytesequenz} $AD$ mittels AEAD
            \begin{itemize}
                \item enthält beide Identitätsinformationen $IK_A, IK_B$ (wie vorher)
                \item ggf. $\text{EncodeKEM}(PQPK_B)$ dranhängen, falls $PQPK_B$ nicht im Ciphertext $CT_{\text{KEM}}$ vearbeitet wird
                \begin{itemize}
                    \item d.h. kodierter Public Key als Bytesequenz
                    \item Warum? $\to$ später
                \end{itemize}

                \item optional: weiter Infos dranhängen
            \end{itemize}

            \item Verschlüsselung der Daten als $CT_{init}$ unter Verwendung von $SK$ und $AD$
        \end{itemize}
    }

    \note<4>{
        \begin{itemize}
            \item Senden der initialen Nachricht an Bob
            \begin{itemize}
                \item Identität von Bob
                \item Alice's Ephemeral Key
                \item KEM-Ciphertext
                \item kodierter Ciphertext
                \item Identifier für verwendete Keys
            \end{itemize}

            \item Abschluss von Phase II.
        \end{itemize}
    }
\end{frame}


\begin{frame}{PQXDH $\blacktriangleright$ III. Empfangen der initialen Nachricht \footnotesize\cite{bhargavanAnalysisSignalsPQXDH2023,kretPQXDHKeyAgreement2024,schmidtAnalysisSignalsPQXDH2024}}
    Bob erhält Nachricht von Alice

    $(IK_A, EK_A, CT_{\text{KEM}}, \alert{CT_{\text{init}}}, \text{ Identifier})$

    $\Rightarrow$ Rückwärtsrechnen von Alice Schritten, Dekodieren statt Enkodieren
\end{frame}


\begin{frame}{PQXDH $\blacktriangleright$ III. Empfangen der initialen Nachricht \footnotesize\cite{bhargavanAnalysisSignalsPQXDH2023,kretPQXDHKeyAgreement2024,schmidtAnalysisSignalsPQXDH2024}}
    \begin{figure}
        \includegraphics[height=6cm]{assets/pqxdh_bob_dh_pqkem.drawio.pdf}
    \end{figure}

    \note{
        \begin{itemize}
            \item Unterschiede zu Alice in rot hervorgehoben
            \item $DH$-Werte analog zu Alice
            \item Dekodieren statt Enkodieren
            \item Input: weiterhin $PQPK_B$ sowie jetzt $CT_{\text{KEM}}$
            \begin{itemize}
                \item war vorher Resultat von $\texttt{PQKEM-ENC}$
            \end{itemize}
            \item wollen Shared Secret $SS_{\text{KEM}}$ berechnen
        \end{itemize}
    }
\end{frame}


\begin{frame}{PQXDH $\blacktriangleright$ III. Empfangen der initialen Nachricht \footnotesize\cite{bhargavanAnalysisSignalsPQXDH2023,kretPQXDHKeyAgreement2024,schmidtAnalysisSignalsPQXDH2024}}
    \only<beamer>{
        % \framezoom(upper left x,upper left y)(zoom area width,zoom area depth)
        \framezoom<0><1>[border](3.6cm,1.75cm)(8.5cm,3.8cm)
        \framezoom<0><2>[border](3.75cm,2.79cm)(6.9cm,2.1cm)
    }

    \begin{figure}
        \includegraphics[height=6cm]{assets/pqxdh_complete_bob.drawio.pdf}
    \end{figure}

    \note<1>{
        \begin{itemize}
            \item Einordnen von DH- und PQKEM-Berechnungen
            \item reinzoomen auf nächster Folie
        \end{itemize}
    }
    \note<2>{
        \begin{itemize}
            \item Wdh. der KDF- und AD-Berechnungen\\
                $\Rightarrow$ Berechnung von Shared Key $SK$ sowie Associated Data $AD$
            \item Verwendung zum Entschlüsseln von $CT_{\text{init}}$
            \item wenn Ciphertext erfolgreich entschlüsselt, dann erfolgreicher Abschluss des Protokolls
        \end{itemize}
    }
\end{frame}


\begin{frame}{PQXDH $\blacktriangleright$ Sicherheitsanalyse \footnotesize\cite{kretPQXDHKeyAgreement2024, schmidtAnalysisSignalsPQXDH2024}}
    % TODO: :Angriffe vorbereiten, erklären können
    \begin{itemize}
        \item formale Verifikation $\to$ Finden von Angriffsmöglichkeiten
        \item \emph{Key Re-Encapsulation} $\to$ $PQPK_B$ bei $AD$-Berechnung
        \begin{itemize}
            \item $AD = \text{EncodeEC}(IK_A) \circ \text{EncodeEC}(IK_B) \circ \alert{\text{EncodeKEM}(PQPK_B)}$
        \end{itemize} 
        \item \emph{Key Compromise} $\to$ häufiger Austausch von Keys, Anpassung Ratchet
    \end{itemize}

    \pause
    \begin{itemize}
        \item kein Schutz vor \emph{Active Quantum Adversaries} (Nicht-Ziel)
        \item PQXDH erfüllt klassische PQ-Sicherheitsanforderungen
        \item weitere Analysen siehe \cite{kretPQXDHKeyAgreement2024,schmidtAnalysisSignalsPQXDH2024}
        \item \cite{schmidtAnalysisSignalsPQXDH2024}: Ratchet muss geschützt werden $\Rightarrow$ \alert{SPQR}
    \end{itemize}

    \note<1>{
        \begin{itemize}
            \item versch. Möglichkeiten in erster Revision gefunden und behoben
            \item Hinzufügen von PQ-Schutz mehr als nur Einbringen von PQ-Krpto, viele \emph{Pitfalls}
            \item hier nur Auswahl an Attacken
        \end{itemize}
    }

    \note<2>{
        \begin{itemize}
            \item Talk über Analyse von PQXDH erwähnt explizit, dass Ratchet geschützt werden müsse \cite{schmidtAnalysisSignalsPQXDH2024}
            \item darüber erzählt Ines euch mehr
        \end{itemize}
    }
\end{frame}
