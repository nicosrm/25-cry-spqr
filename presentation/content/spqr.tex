% !TeX root = ../presentation.tex

\setFooterToRohrbach

\section{Sparse Post Quantum Ratchet (SPQR)}

\begin{frame}{SPQR $\blacktriangleright$ Forward Secrecy und Post-Compromise Security \cite{connellSignalProtocolPostQuantum2025,rolfeschmidtDesigningPostQuantumRatchet2025}}
    \vspace*{0.5cm}
    \begin{itemize}
        % \item Aktueller Stand/Sicherheit soll bei behalten werden
        \item FS und PCS mittels \emph{Continuous Key Agreement} (CKA)
        \begin{itemize}
            \item Verwenden von \emph{Module-Lattice-Based Key-Encapsulation Mechanism} (ML-KEM)
        \end{itemize}
    \end{itemize}

    \pause

    \only{
        \textbf{Diffie-Hellman Ratchet}
        \vspace*{-0.5cm}
        \begin{figure}
            \includegraphics[height=4.5cm]{assets/DH.drawio-2.pdf}
        \end{figure}
    }<1-2>
    \only{
        \textbf{Post Quanten Ratchet}
        \vspace*{-0.9cm}
        \begin{figure}
            \includegraphics[height=4.5cm]{assets/kem.drawio.pdf}
        \end{figure}
    }<3>
    \only{
        \textbf{Post Quanten Ratchet'} % TODO
        \vspace*{-0.9cm}
        \begin{figure}
            \includegraphics[height=4.5cm]{assets/kem.drawio.pdf}
        \end{figure}
    }<4>
\end{frame}

\begin{frame}{SPQR $\blacktriangleright$ Problematik I \cite{connellSignalProtocolPostQuantum2025}}
    \begin{columns}
        \begin{column}{0.5\textwidth}
            \begin{itemize}
                \item ML-KEM basiert auf geordnetem asymmetrischem Austausch
                \pause

                \item[$\Rightarrow$] \emph{State Machine}
            \end{itemize}
        \end{column}
        \pause

        \begin{column}{0.5\textwidth}
            \begin{figure}
                \includegraphics[height=4.5cm]{assets/connellSignalProtocolPostQuantum2025_state.png} % TODO
            \end{figure}
        \end{column}
    \end{columns}
\end{frame}

\begin{frame}{SPQR $\blacktriangleright$ Problematik II \cite{connellSignalProtocolPostQuantum2025,rolfeschmidtDesigningPostQuantumRatchet2025}}
    \textbf{Diffie-Hellman Nachricht}
    \begin{figure}
        \includegraphics[width=\textwidth]{assets/size.pdf} % TODO
    \end{figure}
    \pause

    \textbf{ML-KEM Nachricht (EK/CT)}
    \begin{figure}
        \includegraphics[width=\textwidth]{assets/size.pdf} % TODO
    \end{figure}
    \pause

    \begin{itemize}
        \item Faktor 35
        \item Langsame Performance und hoher Datenverbrauch
        \pause

        \item[$\Rightarrow$] Bandbreiten Limitierung
    \end{itemize}
\end{frame}

\begin{frame}{SPQR $\blacktriangleright$ Bandbreiten Limitierung \cite{auerbachHowCompareBandwidth2025, connellSignalProtocolPostQuantum2025}}
    \begin{itemize}
        
        \item Idee 1: Alle 50 Nachrichten oder einmal pro Woche
        \pause

        \begin{itemize} % TODO
            \item Verwendet von PQ3
            \item Probleme bei asynchroner Kommunikation
            \item Angriffsmöglichkeit $\Rightarrow$ Verhindern neues Geheimnis
        \end{itemize}
        \pause

        \item Idee 2: Austausch aufteilen
        \pause

        \begin{itemize}
            \item[$\Rightarrow$] \emph{Erasure Codes}
        \end{itemize}
    \end{itemize}
\end{frame}

\begin{frame}{SPQR $\blacktriangleright$ Erasure Codes \cite{dodisTripleRatchetBandwidth2025, connellSignalProtocolPostQuantum2025}}
    \begin{itemize} % TODO grafik
        \item Aufteilen von Nachricht $M$ in \emph{Chunks}
        \item alle $n$ Chunks ausreichend um Nachricht zusammenzustellen
        \pause

        \begin{itemize}
            \item[$\Rightarrow$] Nur auffälliger DOS Angriff ab $n-1$ Nachrichten verhindert neues Geheimnis
            \pause

            \item[$\Rightarrow$] Kein CKA $\rightarrow$ \emph{Sparse} CKA (SCKA)
        \end{itemize}
    \end{itemize}
\end{frame}

\begin{frame}{SPQR $\blacktriangleright$ SCKA I \cite{connellSignalProtocolPostQuantum2025}}
    \only{
        \begin{figure}
            \includegraphics[height=6cm]{assets/connellSignalProtocolPostQuantum2025_spqr-1.png} % TODO
        \end{figure}
    }<1>
    \only{
        \begin{figure}
            \includegraphics[height=6cm]{assets/connellSignalProtocolPostQuantum2025_spqr-1.png} % TODO mit "angriffsfläche"
        \end{figure}
    }<2>
\end{frame}

\begin{frame}{SPQR $\blacktriangleright$ SCKA II \cite{connellSignalProtocolPostQuantum2025}}
    \begin{figure}
        \includegraphics[height=6cm]{assets/capacity II.drawio.pdf} % TODO
    \end{figure}
\end{frame}

\begin{frame}{SPQR $\blacktriangleright$ ML-KEM Braid I \cite{connellSignalProtocolPostQuantum2025}}
    \begin{itemize}
        \item A generates an EK of 1184 bytes to send to B, and an associated DK
        \item B receives the EK
        \item B samples a new shared secret (32 bytes), which he encrypts with EK into a CT of 1088 bytes to send to A
        \item A receives the CT, uses the DK to decrypt it, and now also has access to the 32 byte shared secret
    \end{itemize}
    \pause
    \textbf{ML-KEM Braid}
    \begin{itemize}
        \item Aufbrechen in $EK = EK1 + EK2$ und $CT = CT1 + CT2$
    \end{itemize}
\end{frame}

% \begin{frame}{SPQR $\blacktriangleright$ ML-KEM Braid \cite{connellSignalProtocolPostQuantum2025}}
%     \begin{itemize}
%         \item A generates EK and DK. A extracts the 32-byte Seed from EK
%         \item A sends 64 bytes EK1 (Seed + Hash(EK)) to B. B sends nothing during this time.
%         \item B receives the Seed and Hash, and generates the first, largest part of the CT from them (CT1)
%         \item After this point, A sends EK2 (the rest of the EK minus the Seed), while B sends CT1
%         \item B eventually receives EK2, and uses it to generate the final portion of the CT (CT2)
%         \item Once A tells B that she has received all of CT1, B sends A CT2. A sends nothing during this time.
%         \item With both sides having all of the pieces of EK and the CT that they need, they extract their shared secret and increment their epoch
%     \end{itemize}
%     \pause
%     \begin{itemize}
%         \item[$\Rightarrow$] \emph{Sparse Post Quantum Ratchet}
%     \end{itemize}
% \end{frame}

\begin{frame}{SPQR $\blacktriangleright$ ML-KEM Braid \cite{connellSignalProtocolPostQuantum2025}}
    \begin{figure}
        \includegraphics[height=6cm]{assets/connellSignalProtocolPostQuantum2025_spqr-2.png} % TODO
    \end{figure}
\end{frame}

\begin{frame}{SPQR $\blacktriangleright$ State Machine \cite{rolfeschmidtMLKEMBraidProtocol2025}}
    \begin{figure}
        \includegraphics[height=6cm]{assets/connellSignalProtocolPostQuantum2025_spqr-2.png} % TODO
    \end{figure}
\end{frame}

\begin{frame}{Triple Ratchet \cite{connellSignalProtocolPostQuantum2025}}
    \begin{itemize}
        \item Aktuelle Sicherheit behalten
        \begin{itemize}
            \item Hybrid Security
        \end{itemize}
        \item Erweiterung des \emph{Double Ratchet} um SPQR
        \begin{itemize}
            \item Mixen des Outputs mittels \emph{Key Derivation Function} % TODO Grafik
        \end{itemize}
    \end{itemize}
\end{frame}

\begin{frame}{Rollout \cite{connellSignalProtocolPostQuantum2025}}
    \begin{itemize}
        \item Update nicht bei allen zeitgleich möglich $\rightarrow$ keine Kommunikation mehr möglich
        \item Erlauben von \emph{Downgrades}
        \pause

        \begin{itemize}
            \item Mitsenden von SPQR Data ohne Schlüsselmaterial in der Nachricht
            \item Antwort nicht SPQR  $\rightarrow$  mit altem Standard weiter kommunizieren
        \end{itemize}
        \pause

        \item Angriffsmöglichkeiten?
        \begin{itemize}
            \item Entfernen der SPQR nicht möglich
            \item Downgrade nur beim Start möglich
        \end{itemize}
        \pause

        \item Ansatz funktioniert auch für zukünftiges mögliches \emph{SPQRv2}
        \item Irgendwann jede Nachricht mit SPQR geschützt
    \end{itemize}
\end{frame}
