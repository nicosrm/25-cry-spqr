% !TeX root = ../presentation.tex

\setFooterToRohrbach

\section{Sparse Post Quantum Ratchet (SPQR)}

\begin{frame}{SPQR $\blacktriangleright$ Forward Secrecy und Post-Compromise Security \footnotesize\cite{connellSignalProtocolPostQuantum2025,rolfeschmidtDesigningPostQuantumRatchet2025}}
    \vspace*{0.5cm}
    \begin{itemize}
        \item FS und PCS mittels \emph{Continuous Key Agreement} (CKA)
        \begin{itemize}
            \item Verwenden von \emph{Module-Lattice-Based Key-Encapsulation Mechanism} (ML-KEM)
        \end{itemize}
    \end{itemize}
    \pause

    \only{
        \textbf{Diffie-Hellman Ratchet}
        \vspace*{-0.3cm}
        \begin{figure}
            \hspace*{-1.3cm}
            \includegraphics[height=3.9cm]{assets/dh.drawio.pdf}
            \vspace*{-0.4cm}
            \caption{DHR Schlüsselgenerierung (orientiert an \cite{rolfeschmidtDesigningPostQuantumRatchet2025})}
        \end{figure}
    }<1-2>
    \only{
        \textbf{Post Quanten Ratchet}
        \vspace*{-0.3cm}
        \begin{figure}
            \hspace*{-1.3cm}
            \includegraphics[height=3.9cm]{assets/kem.drawio.pdf}
            \vspace*{-0.4cm}
            \caption{PQR Schlüsselgenerierung (orientiert an \cite{rolfeschmidtDesigningPostQuantumRatchet2025})}
        \end{figure}
    }<3>
    \only{
        \textbf{Post Quanten Ratchet\alert{*}}
        \vspace*{-0.3cm}
        \begin{figure}
            \hspace*{-1.3cm}
            \includegraphics[height=3.9cm]{assets/kem-s.drawio.pdf}
            \vspace*{-0.4cm}
            \caption{PQR* Schlüsselgenerierung}
        \end{figure}
    }<4>
\end{frame}

\begin{frame}{SPQR $\blacktriangleright$ Problematik I \footnotesize\cite{connellSignalProtocolPostQuantum2025}}
    \begin{columns}
        \begin{column}{0.4\textwidth}
            \begin{itemize}
                \item ML-KEM basiert auf geordnetem asymmetrischem Austausch
                \pause

                \begin{itemize}
                    \item[$\Rightarrow$] \emph{State Machine}
                \end{itemize}
            \end{itemize}
        \end{column}
        \pause

        \begin{column}{0.6\textwidth}
            \begin{figure}
                \includegraphics[height=4.5cm]{assets/sm_cka.drawio.pdf}
                \caption{\emph{State Machine} von CKA (orientiert an \cite{connellSignalProtocolPostQuantum2025})}
            \end{figure}
        \end{column}
    \end{columns}
\end{frame}

\begin{frame}{SPQR $\blacktriangleright$ Problematik II \footnotesize\cite{connellSignalProtocolPostQuantum2025,rolfeschmidtDesigningPostQuantumRatchet2025}}
    \vspace*{0.3cm}
    \textbf{Diffie-Hellman Nachricht}
    \vspace*{-0.5cm}
    \begin{figure}
        \hspace*{-1cm}
        \includegraphics[width=15.5cm]{assets/dh_m.drawio.pdf}
    \end{figure}
    \pause

    \textbf{ML-KEM Nachricht (EK/CT)}
    \vspace*{-0.5cm}
    \begin{figure}
        \hspace*{-1cm}
        \includegraphics[width=15.5cm]{assets/kem_m.drawio.pdf}
    \end{figure}
    \pause

    \begin{itemize}
        \item Faktor \alert{35}
        \item Langsame Performance und hoher Datenverbrauch
        \pause

        \item[$\Rightarrow$] Bandbreiten Limitierung
    \end{itemize}
\end{frame}

\begin{frame}{SPQR $\blacktriangleright$ Bandbreiten Limitierung \footnotesize\cite{auerbachHowCompareBandwidth2025, connellSignalProtocolPostQuantum2025,dodisTripleRatchetBandwidth2025}}
    \begin{itemize}
        
        \item Idee 1: Alle 50 Nachrichten oder einmal pro Woche
        \pause

        \begin{itemize} % TODO
            \item Probleme bei asynchroner Kommunikation
            \item Angriffsmöglichkeit $\Rightarrow$ Verhindern neues Geheimnis
            \item Umgesetzt von PQ3
        \end{itemize}
        \pause

        \item Idee 2: Austausch aufteilen
        \pause

        \begin{itemize}
            \item[$\Rightarrow$] \emph{Erasure Codes}
        \end{itemize}
    \end{itemize}
\end{frame}

\begin{frame}{SPQR $\blacktriangleright$ Erasure Codes \footnotesize\cite{dodisTripleRatchetBandwidth2025, connellSignalProtocolPostQuantum2025,rolfeschmidtDesigningPostQuantumRatchet2025}}
    \vspace*{0.2cm}
    \begin{figure}
        \includegraphics[height=4cm]{assets/erasure.drawio.pdf}
        \caption{Chunking mit Erasure Codes (orientiert an \cite{rolfeschmidtDesigningPostQuantumRatchet2025})}
    \end{figure}
    \pause

    \vspace*{-0.3cm}
    \begin{itemize}
        \item[$\Rightarrow$] Nur auffälliger DOS Angriff ab $n-1$ Nachrichten verhindert neues Geheimnis
        \pause

        \item[$\Rightarrow$] Kein CKA $\rightarrow$ \emph{Sparse} CKA (SCKA)
    \end{itemize}
\end{frame}

\begin{frame}{SPQR $\blacktriangleright$ SCKA I \footnotesize\cite{connellSignalProtocolPostQuantum2025}}
    \begin{figure}
        \only{
            \includegraphics[height=6cm]{assets/scka_1.drawio.pdf}
        }<1|handout:0>
        \only{
            \includegraphics[height=6cm]{assets/scka_2.drawio.pdf}
        }<2|handout:0>
        \only{
            \includegraphics[height=6cm]{assets/scka_3.drawio.pdf}
        }<3>
        
        \vspace*{-0.3cm}
        \caption{Kommunikation mit \emph{Sparse Continuous Key Agreement} (orientiert an \cite{connellSignalProtocolPostQuantum2025,rolfeschmidtDesigningPostQuantumRatchet2025})}
    \end{figure}
\end{frame}

\begin{frame}{SPQR $\blacktriangleright$ SCKA II \footnotesize\cite{connellSignalProtocolPostQuantum2025}}
    \begin{figure}
        \includegraphics[height=6cm]{assets/scka 2.drawio.pdf}
        \caption{SCKA mit vielen generierten Geheimnissen}
    \end{figure}
\end{frame}

\begin{frame}{SPQR $\blacktriangleright$ ML-KEM \footnotesize\cite{connellSignalProtocolPostQuantum2025,rolfeschmidtDesigningPostQuantumRatchet2025}}
    \begin{columns}
        \begin{column}{0.5\textwidth}
            Bestandteile \emph{Encapsulation Key} (EK):
            \begin{itemize}
                \item Seed, welche zur Matrix $A$ expandiert werden kann
                \item Vektor $As+e$
            \end{itemize}

            \vspace*{0.5cm}
            Bestandteile \emph{Ciphertext} (CT):
            \begin{itemize}
                \item Vektor $A^Ts'+e'$
                \item Nachricht für den Abgleich
            \end{itemize}
        \end{column}

        \begin{column}{0.5\textwidth}
            \begin{figure}
                \hspace*{-0.3cm}
                \includegraphics[height=5cm]{assets/kem_p.drawio.pdf}
                \caption{\emph{Encapsulation Key} und \emph{Ciphertext} (orientiert an \cite{rolfeschmidtDesigningPostQuantumRatchet2025})}
            \end{figure}
        \end{column}
    \end{columns}
\end{frame}

\begin{frame}{SPQR $\blacktriangleright$ ML-KEM Braid \footnotesize\cite{connellSignalProtocolPostQuantum2025,rolfeschmidtDesigningPostQuantumRatchet2025}}
    \begin{figure}
        \only{
            \includegraphics[height=6cm]{assets/spqr_1.drawio.pdf}
        }<1|handout:0>
        \only{
            \includegraphics[height=6cm]{assets/spqr_2.drawio.pdf}
        }<2|handout:0>
        \only{
            \includegraphics[height=6cm]{assets/spqr.drawio.pdf}
        }<3>

        \caption{ML-KEM Braid Ablauf (orientiert an \cite{connellSignalProtocolPostQuantum2025,rolfeschmidtDesigningPostQuantumRatchet2025})}
    
    \end{figure}
\end{frame}

\begin{frame}{SPQR $\blacktriangleright$ State Machine \footnotesize\cite{rolfeschmidtMLKEMBraidProtocol2025}}
    \begin{figure}
        \includegraphics[height=6cm]{assets/sm_spqr.drawio.pdf}
        \caption{\emph{State Machine} von ML-KEM Braid (orientiert an \cite{rolfeschmidtMLKEMBraidProtocol2025})}
    \end{figure}
\end{frame}

\begin{frame}{Triple Ratchet \footnotesize\cite{connellSignalProtocolPostQuantum2025}}
    \begin{columns}
        \begin{column}{0.4\textwidth}
            \begin{itemize}
            \item Aktuelle Sicherheit behalten
            \item Erweiterung der \emph{Double Ratchet} (DR) um SPQR
            \item Codeänderung minimal
        \end{itemize}
        \end{column}

        \begin{column}{0.6\textwidth}
            \begin{figure}
                \hspace*{-0.3cm}
                \includegraphics[height=5.5cm]{assets/tr.drawio.pdf}
                \caption{Triple Ratchet (orientiert an \cite{rolfeschmidtDesigningPostQuantumRatchet2025})}
            \end{figure}
        \end{column}
    \end{columns}
\end{frame}

\begin{frame}{Rollout \footnotesize\cite{connellSignalProtocolPostQuantum2025}}
    \begin{itemize}
        \item Update nicht bei allen zeitgleich möglich $\rightarrow$ keine Kommunikation mehr möglich
        \item Erlauben von \emph{Downgrades}
        \pause

        \begin{itemize}
            \item Mitsenden von \emph{SPQR Data} in der Nachricht
            \item Falls Antwort nicht SPQR  $\rightarrow$ mit DR kommunizieren
        \end{itemize}
        \pause

        \item Angriffsmöglichkeiten?
        \begin{itemize}
            \item Entfernen der SPQR?
            \item Downgrade später erzwingen?
        \end{itemize}
        \pause

        \item Ansatz funktioniert auch für zukünftiges mögliches \emph{SPQRv2}
        \item Irgendwann garantiert \alert{jede} Nachricht mit SPQR geschützt
    \end{itemize}
\end{frame}
