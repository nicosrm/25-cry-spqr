% !TeX root = ../presentation.tex

\setFooterToRohrbach

\section{Sparse Post Quantum Ratchet (SPQR)}

\begin{frame}{SPQR $\blacktriangleright$ Forward Secrecy und Post-Compromise Security \cite{connellSignalProtocolPostQuantum2025}}
    \begin{itemize}
        \item Aktueller Stand soll bei behalten werden
        \item FS und PCS mittels CKA
        \begin{itemize}
            \item Verwenden von ML-KEM
        \end{itemize}
    \end{itemize}
\end{frame}

\begin{frame}{SPQR $\blacktriangleright$ Bandbreiten Limitierung \cite{auerbachHowCompareBandwidth2025, connellSignalProtocolPostQuantum2025}}
    \begin{itemize}
        \item Idee 1: Wie aktuell mit jeder Nachricht neu vereinbaren
        % Grafik für 71x so groß
        \begin{itemize}
            \item Nachricht zu groß $\Rightarrow$ langsame Performance und hoher Datenverbrauch
        \end{itemize}
        \item Idee 2: Alle 50 Nachrichten oder einmal pro Woche
        \begin{itemize}
            \item Verwendet von PQ3
            \item Probleme bei asynchroner Kommunikation
            \item Angriffsmöglichkeit $\Rightarrow$ Verhindern neues Geheimnis
        \end{itemize}
        \item Idee 3: Austausch aufteilen
        \begin{itemize}
            \item[$\Rightarrow$] \emph{Erasure Codes}
        \end{itemize}
    \end{itemize}
\end{frame}

\begin{frame}{SPQR $\blacktriangleright$ Erasure Codes \cite{dodisTripleRatchetBandwidth2025, connellSignalProtocolPostQuantum2025}}
    \begin{itemize}
        \item Aufteilen von Nachricht $M$ in \emph{Chunks}
        \item alle $n$ Chunks ausreichend um Nachricht zusammenzustellen
        \begin{itemize}
            \item[$\Rightarrow$] Nur auffälliger DOS Angriff ab $n-1$ Nachrichten verhindert neues Geheimnis
        \end{itemize}
        % \item An erasure code for a set of symbols Σ, a block length N , and a message size nchunk consists of PPT algorithms Encode, Decode defined as follows: Encode(M, i) → c : It takes as input a message M ∈ Σnchunk , and an integer i ∈ ZN and outputs symbol c ∈ Σ. Decode(L) → M : It takes as input a set L ⊂ ZN × Σ and outputs either a message M ∈ Σnchunk or the symbol ⊥. An erasure code is said to be correct if for all messages M ∈ Σnchunk , for all I ⊂ ZN and L = {(i, Encode(M, i) | i ∈ I)} we have that Decode(L, nchunk) = M if |I| = nchunk and Decode(L, nchunk) = ⊥ if |I| < nchunk.
        % \item \emph{Reed-Solomon erasure codes}
    \end{itemize}
\end{frame}

\begin{frame}{SPQR $\blacktriangleright$ Kapazität I \cite{connellSignalProtocolPostQuantum2025}}
    \begin{figure}
        \includegraphics[height=6cm]{assets/connellSignalProtocolPostQuantum2025_spqr-1.png}
    \end{figure}
\end{frame}

\begin{frame}{SPQR $\blacktriangleright$ Kapazität II \cite{connellSignalProtocolPostQuantum2025}}
    \begin{figure}
        \includegraphics[height=6cm]{assets/capacity II.drawio.pdf}
    \end{figure}
\end{frame}

\begin{frame}{SPQR $\blacktriangleright$ Sparse Post Quantum Ratchet \cite{connellSignalProtocolPostQuantum2025}}
    \begin{itemize}
        \item ML-KEM
        \begin{itemize}
            \item Aufbrechen in $EK = EK1 + EK2$ und $CT = CT1 + CT$
            \item[$\Rightarrow$] ML-KEM Braid
        \end{itemize}
        \item[$\Rightarrow$] \emph{Sparse Post Quantum Ratchet}
    \end{itemize}
\end{frame}

\begin{frame}{SPQR $\blacktriangleright$ Sparse Post Quantum Ratchet \cite{connellSignalProtocolPostQuantum2025}}
    \begin{figure}
        \includegraphics[height=6cm]{assets/connellSignalProtocolPostQuantum2025_spqr-2.png}
    \end{figure}
\end{frame}

\begin{frame}{Triple Ratchet \cite{connellSignalProtocolPostQuantum2025}}
    \begin{itemize}
        \item Aktuelle Sicherheit behalten
        \begin{itemize}
            \item Hybrid Security
        \end{itemize}
        \item Erweiterung des \emph{Double Ratchet} um SPQR
        \begin{itemize}
            \item Mixen des Outputs mittels \emph{Key Derivation Function}
        \end{itemize}
    \end{itemize}
\end{frame}

\begin{frame}{Rollout \cite{connellSignalProtocolPostQuantum2025}}
    \begin{itemize}
        \item Update nicht bei allen zeitgleich möglich $\rightarrow$ keine Kommunikation mehr möglich
        \item Erlauben von \emph{Downgrades}
        \begin{itemize}
            \item Mitsenden von SPQR Data ohne Schlüsselmaterial in der Nachricht
            \item Antwort nicht SPQR  $\rightarrow$  mit altem Standard weiter kommunizieren
        \end{itemize}
        \item Angriffsmöglichkeit?
        \begin{itemize}
            \item Entfernen der SPQR nicht möglich
            \item Downgrade nur beim Start möglich
        \end{itemize}
        \item Ansatz funktioniert auch für zukünftiges mögliches \emph{SPQRv2}
        \item Irgendwann jede Nachricht mit SPQR geschützt
    \end{itemize}
\end{frame}
