% !TeX root = ../presentation.tex

\setbeamertemplate{frame footer}{
  \tiny Mager, Rohrbach, Schramm: \insertshorttitle, \insertdate\newline\module, \prof, \semester}

{
    \metroset{sectionpage=none}
    \section{Einleitung}
}

\begin{frame}{Einleitung}
    \begin{itemize}
        \item Abholen von vorherigem Vortrag
        \item Warum Post-Quantum-Schutz?
        \item Motivation durch zukünftige Angriffe
    \end{itemize}
\end{frame}


\begin{frame}{Ablauf Chat}
    \begin{itemize}
        \item Initialer Handshake (X3DH $\rightarrow$ PQXDH)
        \item Nachrichten senden (SPQR)
    \end{itemize}
\end{frame}


\begin{frame}{Quantencomputing – Grundidee}
    \begin{itemize}
        \item Qubits statt klassischer Bits
        \item Superposition: gleichzeitiger Zustand von 0 und 1
        \item Verschränkung zwischen Qubits
        \item Gemeinsame Operationen auf vielen Zuständen
        \item Interferenz verstärkt korrekte Ergebnisse
    \end{itemize}
\end{frame}


\begin{frame}{Bedeutung für die Kryptografie}
    \begin{itemize}
        \item RSA und ECC gelten heute als sicher
        \item Sicherheit basiert auf hohem Rechenaufwand
        \item Quantenalgorithmen sind asymptotisch schneller
        \item Bisher unlösbare Probleme werden lösbar
        \item Durchbruch abhängig von Rechnergröße
    \end{itemize}
\end{frame}


\begin{frame}{Aktueller Stand der Technik}
    \begin{itemize}
        \item Spezielle Quantenprogramme bereits schneller
        \item Geschwindigkeitsvorteile bis zu $13{.}000\times$
        \item Experimentelle Demonstrationen existieren
        \item Noch keine praktischen Angriffe auf Kryptosysteme
        \item Quantenhardware aktuell stark limitiert
    \end{itemize}
\end{frame}


\begin{frame}{Algorithmischer Kern \& Ausblick}
    \begin{itemize}
        \item Shors Algorithmus als zentrale Bedrohung
        \item Reduktion auf Periodenfindung
        \item Nutzung von Quanten-Parallelität
        \item Sehr hoher Bedarf an fehlerkorrigierten Qubits
        \item Post-Quantum-Kryptografie langfristig notwendig
    \end{itemize}
\end{frame}


\begin{frame}{Einführung von Quanten-Sicherheit in Signal \footnotesize\cite{connellSignalProtocolPostQuantum2025, gentyalaQuantumResistanceSignal2025}}
    \begin{itemize}
        \item[I.] Absicherung des initialen Handshakes (X3DH) mit \alert{PQXDH}
        \begin{itemize}
            \item Schutz vor \emph{Harvest Now, Decrypt Later}-Angriffen
            \item Hybrider Ansatz: Elliptic Curve + \alert{CRYSTALS-Kyber}
        \end{itemize}

        \item[II.] Verbesserung von \emph{Forward Secrecy} und \emph{Post-Compromise Security}
        \begin{itemize}
            \item \alert{SPQR}: Double Ratchet $\rightarrow$ Triple Ratchet
        \end{itemize}
    \end{itemize}
\end{frame}
