% !TeX root = ../presentation.tex

\begin{frame}<1|handout:0>{Key Re-Encapsulation Attack \footnotesize\cite{bhargavanAnalysisSignalsPQXDH2023,schmidtAnalysisSignalsPQXDH2024}}
    \begin{figure}
        \includegraphics[width=\textwidth]{assets/schmidt_pqxdh_kem_reencapsulation_attack.png}
        \caption{Key Re-Encapsulation Attack \cite{schmidtAnalysisSignalsPQXDH2024}}
    \end{figure}
    \vspace{-0.8cm}

    \note{
        \begin{itemize}
            \item Kompromittierung \emph{eines} $PQPK$ alle anderen $PQPK'$ auch kompromittiert
            
            \item O\&E kompromittieren $PQPK_B^1$, beziehen $CT$, lassen $A$ verschlüsseln, erfahren $SS$
            \item O\&E finden neuen $CT'$, welcher gültig für $PQPK_B^2$ ist, sodass $\texttt{DEC}(CT') = SS$
            \item O\&E leiten Nachricht von A nach B weiter, aber tauschen $CT$ mit $CT^{\text{comp}}$ und Key Identifier von $PQPK$ zu $PQPK_B^1$ aus
            \item B kann Key ebenfalls ausrechnen unter Verwendung von $PQPK_B^2$
        \end{itemize}
    }
\end{frame}
