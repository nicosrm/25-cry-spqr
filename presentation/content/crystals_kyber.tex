% !TeX root = ../presentation.tex

\setFooterToMager

\section{CRYSTALS-Kyber}

\begin{frame}{CRYSTALS-Kyber}
    \begin{columns}
        \begin{column}{0.5\textwidth}
            \begin{itemize}
                \item<1-> Key Encapsulation Mechanism (KEM)
                \item<2-> Empfehlung: hybrider Einsatz mit Prä-Quanten-Verfahren (z.\,B. ECC)
                \item<3-> Basiert auf \emph{Learning With Errors} (LWE) auf Modul-Gittern
            \end{itemize}
        \end{column}

        \begin{column}{0.5\textwidth}
            \centering
            \includegraphics[scale=0.8]{assets/KEMvsPKE.pdf}

            \vspace{0.5em}
            {\small \textbf{KEM vs PKE} \footnotesize\cite{kemDiagram2026}}
        \end{column}
    \end{columns}
\end{frame}


\begin{frame}{CRYSTALS-Kyber $\blacktriangleright$ Learning With Errors: Grundidee}
\[
\begin{pmatrix}
14 & 15 & 5  & 2 \\
13 & 14 & 14 & 6 \\
6  & 10 & 13 & 1 \\
10 & 4  & 12 & 16
\end{pmatrix}
\cdot
\begin{pmatrix}
s_1 \\
s_2 \\
s_3 \\
s_4
\end{pmatrix}
=
\begin{pmatrix}
262 \\
374 \\
258 \\
336
\end{pmatrix}
\]
\end{frame}


\begin{frame}{CRYSTALS-Kyber $\blacktriangleright$ Learning With Errors: Formalisierung}
\[
A =
\begin{pmatrix}
14 & 15 & 5  & 2  \\
13 & 14 & 14 & 6  \\
6  & 10 & 13 & 1  \\
10 &  4 & 12 & 16 \\
9  &  5 & 9  & 6  \\
3  &  6 & 4  & 5  \\
6  &  7 & 17 & 2
\end{pmatrix}, \quad
s =
\begin{pmatrix}
0 \\ 13 \\ 9 \\ 11
\end{pmatrix}, \quad
e =
\begin{pmatrix}
1 \\ -1 \\ 0 \\ -1 \\ 1 \\ 0 \\ 1
\end{pmatrix}
\]

\[
q = 17, \quad b = (A \cdot s + e) \bmod q = 
\begin{pmatrix} 8 & 16 & 3 & 12 & 9 & 16 & 3 \end{pmatrix}^T
\]
\end{frame}


\begin{frame}{CRYSTALS-Kyber $\blacktriangleright$ Kleines Beispiel \footnotesize\cite{Gonzalez2021Kyber}}
    \begin{itemize}
        \item Modulus: $q = 17$
        \item Polynomring: $\mathbb{Z}_{17}[X]/(X^4 + 1)$
        \item Privater Schlüssel:
        \[
        s = (-x^3 - x^2 + x,\; -x^3 - x)
        \]
    \end{itemize}
\end{frame}


\begin{frame}{CRYSTALS-Kyber $\blacktriangleright$ Public Key}
    \begin{itemize}
        \item Public Key: $(A, t)$ mit $t = A s + e$
    \end{itemize}

\[
A =
\begin{pmatrix}
6x^3 + 16x^2 + 16x + 1 & 15x^3 + 3x^2 + 10x + 1 \\
9x^3 + 4x^2 + 6x + 3  & 6x^3 + x^2 + 9x + 15
\end{pmatrix}
\]

\[
e = (x^2,\; x^2 - x)
\]

\[
t = (16x^3 + 15x^2 + 7,\; 10x^3 + 12x^2 + 11x + 6)
\]
\end{frame}


\begin{frame}{CRYSTALS-Kyber $\blacktriangleright$ Verschlüsseln}
\[
u = A^T r + e_1
\qquad
v = t^T r + e_2 + m
\qquad
c = (u,v)
\]

\textbf{Beispielrechnung:}
\[
r = (-x^3 + x^2,\; x^3 + x^2 - 1)
\]
\[
e_1 = (x^2 + x,\; x^2)
\qquad
e_2 = -x^3 - x^2
\]
\[
m_b = 11_{10} = 1011_2 = x^3 + x + 1
\]
\[
m = \left\lfloor \frac{q}{2} \right\rceil \cdot m_b = 9x^3 + 9x + 9
\]
\[
u = (11x^3 + 11x^2 + 10x + 3,\; 4x^3 + 4x^2 + 13x + 11)
\]
\[
v = 7x^3 + 6x^2 + 8x + 15
\]
\end{frame}


\begin{frame}{CRYSTALS-Kyber $\blacktriangleright$ Entschlüsseln}
\[
m_n = v - s^T u
\]
\[
m_n = e^T r + e_2 + m + s^T e_1
\]
\begin{itemize}
    \item Rauschen ist enthalten
    \item Rundung auf $\left\lfloor \frac{q}{2} \right\rceil$ oder $q$
\end{itemize}
\[
m_b = \frac{1}{9}(9x^3 + 0x^2 + 9x + 9)
      = x^3 + x + 1
\]
\[
m_b = (1011)_2 = (11)_{10}
\]
\end{frame}


\begin{frame}{CRYSTALS-Kyber $\blacktriangleright$ Parameter}

\begin{tabular}{|l|c|c|c|c|c|c|c|c|}
\hline
Name & $n$ & $k$ & $q$ & $\eta_1$ & $\eta_2$ & $d_u$ & $d_v$ & $\delta$ \\
\hline\hline
Baby-Kyber & 4 & 2 & 17 & 1 & -1 & 1 & 1 & $2^{-4}$ \\
Kyber512  & 256 & 2 & 3329 & 3 & 2 & 10 & 4 & $2^{-139}$ \\
Kyber768  & 256 & 3 & 3329 & 2 & 2 & 10 & 4 & $2^{-164}$ \\
Kyber1024 & 256 & 4 & 3329 & 2 & 2 & 11 & 5 & $2^{-174}$ \\
\hline
\end{tabular}

\vspace{0.5em}

\begin{columns}
    \begin{column}{0.5\textwidth}
        \small
        \begin{itemize}
            \item $n$: maximaler Grad der Polynome
            \item $k$: Anzahl der Polynome pro Vektor
            \item $q$: Modulus
        \end{itemize}
    \end{column}
    \begin{column}{0.5\textwidth}
        \small
        \begin{itemize}
            \item $\eta_1, \eta_2$: Begrenzen die Größe der Fehlerterme
            \item $d_u, d_v$: Kompressionsparameter
            \item$\delta$: Wahrscheinlichkeit einer Fehlentschlüsselung
        \end{itemize}
    \end{column}
\end{columns}
\end{frame}

