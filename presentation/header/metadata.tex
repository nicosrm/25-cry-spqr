% !TeX root = ../presentation.tex

\title[Post-Quanten-Kryptografie in Signal]{Post-Quanten-Kryptografie in Signal}

% \subtitle[SUBTITLE]
%     {FORMATTED\\SUBTITLE}

\date{19. Januar 2026}
\author[Mager, Rohrbach, Schramm]{Max Mager, Ines Rohrbach, Nico Schramm}

\titlegraphic{\flushright\includegraphics[width=2cm]{./assets/htwk_logo.png}}

\newcommand{\module}{C179 Kryptologie}
\newcommand{\prof}{Prof. Dr. Martin Grüttmüller}

\newcommand{\faculty}{Fakultät Informatik und Medien}
\newcommand{\university}{Hochschule für Technik, Wirtschaft und Kultur Leipzig}
\newcommand{\semester}{WiSe 2025/26}

\newcommand{\repourl}{https://github.com/nicosrm/25-cry-spqr}
\newcommand{\githubHandle}{nicosrm/25-cry-spqr}

\institute{%
    \setlength\parindent{-6pt}
    \begin{tabular}{ll}
        \module              & \faculty\\
        \prof                & \university\\[2pt]
        \footnotesize\ccbysa & \footnotesize\href{\repourl}{\faicon{github} \githubHandle}
    \end{tabular}
    \vspace{-1em}
}

\hypersetup{
    % PDF metadata
    pdfauthor={Max Mager, Ines Rohrbach, Nico Schramm},
    pdftitle={Post-Quanten-Kryptografie in Signal},
    pdfsubject={Präsentation},
    % hyperref colouring
    colorlinks,
    allcolors=.,
    urlcolor=blue,
    citecolor=cyan
}
